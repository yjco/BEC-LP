\documentclass[12pt]{article}

% packages ---

\usepackage{tikz}
\usetikzlibrary{arrows}
\usetikzlibrary{plotmarks}
\usetikzlibrary{positioning}
\usetikzlibrary{backgrounds}
\usetikzlibrary{decorations.markings}

\usepackage{pgfplots}
\usepgfplotslibrary{colormaps}
%\usepgfplotslibrary{external}

\usepackage{xfp}
\usepackage{float}
\usepackage{color}
%\usepackage{times}
\usepackage{empheq}
\usepackage{comment}
\usepackage{caption}
\usepackage{xstring}
\usepackage{amsmath}
\usepackage{amssymb}
\usepackage{sectsty}
\usepackage{animate}
\usepackage{listings}
\usepackage{titlesec}
\usepackage{hyperref}
\usepackage{graphicx}
\usepackage{geometry}
\usepackage{setspace}
\usepackage{subcaption}
\usepackage{indentfirst}

\usepackage[portuguese]{babel}

% packages ---


% configuration ---

\geometry{a4paper, left=3cm, right=2cm, top=2.5cm, bottom=2.5cm}
\onehalfspacing

\sectionfont{\normalsize}
\subsectionfont{\normalsize}
\titlelabel{\thetitle.\quad}

\DeclareCaptionType{equ}[Equation][]
\newcommand*\widefbox[1]{\fbox{\hspace{2em}#1\hspace{2em}}}

\title{\textbf{Prova I}\\\large Estrutura de Dados\\\normalsize Professor Eduardo Takeo Ueda\\ Senac Santo Amaro}
\author{Yuri Jorge Carrião Otofuji}
\date{01/10/2021}

\DeclareCaptionLabelSeparator{traço}{ - }
\captionsetup{
   labelsep=traço,
   font={ stretch=2 }
}

% HyperLink SetUp
\hypersetup{
    colorlinks=true,
    linkcolor=black,
    urlcolor=blue
}

%\tikzexternalize

\lstdefinestyle{customc}{
  belowcaptionskip=1\baselineskip,
  breaklines=true,
  frame=L,
  xleftmargin=\parindent,
  language=C,
  showstringspaces=false,
  basicstyle=\footnotesize\ttfamily,
  keywordstyle=\bfseries\color{green!40!black},
  commentstyle=\itshape\color{purple!40!black},
  identifierstyle=\color{blue},
  stringstyle=\color{orange},
}

\lstdefinestyle{customasm}{
  belowcaptionskip=1\baselineskip,
  frame=L,
  xleftmargin=\parindent,
  language=[x86masm]Assembler,
  basicstyle=\footnotesize\ttfamily,
  commentstyle=\itshape\color{purple!40!black},
}

\lstset{escapechar=@,style=customc}

% configuration ---


\begin{document}

	% Cover
	\pagenumbering{gobble}
	\maketitle
	
	% Contents
	\tableofcontents
	
	\newpage
	
	% Defining page numbering format
	\pagenumbering{roman}
	%\rfoot{\thepage}

	\section{Implemente (em linguagem C) um algoritmo recursivo que inverta a ordem dos elementos de um vetor de inteiros.}
	\lstinputlisting[style=customc]{1.c}

	\newpage

	\section{Usando a definição de notação O, prove que $log_{10}n$ é $O(lg(n))$ . (Lembre-se que $lg(n)$ denota o logaritmo de n na base 2).}
	\begin{center}
		$log(n) \leq clg(n)$ \\
		$log(n) \leq c\frac{log(n)}{log(2)}$ \\
		$n_0=2 \quad c=log(2)$
	\end{center}

	\newpage

	\section{Implemente (em linguagem C) uma função que busca/procura, recursivamente, um valor em uma lista simplesmente encadeada/ligada.}
	\lstinputlisting[style=customc]{3.c}

	\newpage

	\section{Implemente (em linguagem C), utilizando uma pilha, uma função que recebe uma sequência de caracteres e verifica se ela é um palíndromo, ou seja, se a \textit{string} é escrita da mesma maneira de frente para trás e de trás para frente (ignore espaços e pontos).}
	\lstinputlisting[style=customc]{5.c}

	\newpage

	\section{Implemente (em linguagem C) uma fila utilizando duas pilhas. A fila deve realizar as operações usuais de enfileirar (\textit{enqueue}) e desenfileirar (\textit{dequeue}) elementos, assim como as pilhas devem efetuar as operações de empilhar (\textit{push}) e desempilhar (\textit{pop}).}
	\lstinputlisting[style=customc]{5.c}

\end{document}
